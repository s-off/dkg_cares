% Options for packages loaded elsewhere
\PassOptionsToPackage{unicode}{hyperref}
\PassOptionsToPackage{hyphens}{url}
%
\documentclass[
]{article}
\usepackage{amsmath,amssymb}
\usepackage{lmodern}
\usepackage{iftex}
\ifPDFTeX
  \usepackage[T1]{fontenc}
  \usepackage[utf8]{inputenc}
  \usepackage{textcomp} % provide euro and other symbols
\else % if luatex or xetex
  \usepackage{unicode-math}
  \defaultfontfeatures{Scale=MatchLowercase}
  \defaultfontfeatures[\rmfamily]{Ligatures=TeX,Scale=1}
\fi
% Use upquote if available, for straight quotes in verbatim environments
\IfFileExists{upquote.sty}{\usepackage{upquote}}{}
\IfFileExists{microtype.sty}{% use microtype if available
  \usepackage[]{microtype}
  \UseMicrotypeSet[protrusion]{basicmath} % disable protrusion for tt fonts
}{}
\makeatletter
\@ifundefined{KOMAClassName}{% if non-KOMA class
  \IfFileExists{parskip.sty}{%
    \usepackage{parskip}
  }{% else
    \setlength{\parindent}{0pt}
    \setlength{\parskip}{6pt plus 2pt minus 1pt}}
}{% if KOMA class
  \KOMAoptions{parskip=half}}
\makeatother
\usepackage{xcolor}
\IfFileExists{xurl.sty}{\usepackage{xurl}}{} % add URL line breaks if available
\IfFileExists{bookmark.sty}{\usepackage{bookmark}}{\usepackage{hyperref}}
\hypersetup{
  pdftitle={Rehabilitation nach Tumorerkrankungen: Sekundärdatenanalyse zur Rückkehr von Krebspatient:innen in den Beruf, 2010--2017},
  pdfauthor={Johannes Soff},
  hidelinks,
  pdfcreator={LaTeX via pandoc}}
\urlstyle{same} % disable monospaced font for URLs
\usepackage[margin=1in]{geometry}
\usepackage{longtable,booktabs,array}
\usepackage{calc} % for calculating minipage widths
% Correct order of tables after \paragraph or \subparagraph
\usepackage{etoolbox}
\makeatletter
\patchcmd\longtable{\par}{\if@noskipsec\mbox{}\fi\par}{}{}
\makeatother
% Allow footnotes in longtable head/foot
\IfFileExists{footnotehyper.sty}{\usepackage{footnotehyper}}{\usepackage{footnote}}
\makesavenoteenv{longtable}
\usepackage{graphicx}
\makeatletter
\def\maxwidth{\ifdim\Gin@nat@width>\linewidth\linewidth\else\Gin@nat@width\fi}
\def\maxheight{\ifdim\Gin@nat@height>\textheight\textheight\else\Gin@nat@height\fi}
\makeatother
% Scale images if necessary, so that they will not overflow the page
% margins by default, and it is still possible to overwrite the defaults
% using explicit options in \includegraphics[width, height, ...]{}
\setkeys{Gin}{width=\maxwidth,height=\maxheight,keepaspectratio}
% Set default figure placement to htbp
\makeatletter
\def\fps@figure{htbp}
\makeatother
\setlength{\emergencystretch}{3em} % prevent overfull lines
\providecommand{\tightlist}{%
  \setlength{\itemsep}{0pt}\setlength{\parskip}{0pt}}
\setcounter{secnumdepth}{-\maxdimen} % remove section numbering
\newlength{\cslhangindent}
\setlength{\cslhangindent}{1.5em}
\newlength{\csllabelwidth}
\setlength{\csllabelwidth}{3em}
\newlength{\cslentryspacingunit} % times entry-spacing
\setlength{\cslentryspacingunit}{\parskip}
\newenvironment{CSLReferences}[2] % #1 hanging-ident, #2 entry spacing
 {% don't indent paragraphs
  \setlength{\parindent}{0pt}
  % turn on hanging indent if param 1 is 1
  \ifodd #1
  \let\oldpar\par
  \def\par{\hangindent=\cslhangindent\oldpar}
  \fi
  % set entry spacing
  \setlength{\parskip}{#2\cslentryspacingunit}
 }%
 {}
\usepackage{calc}
\newcommand{\CSLBlock}[1]{#1\hfill\break}
\newcommand{\CSLLeftMargin}[1]{\parbox[t]{\csllabelwidth}{#1}}
\newcommand{\CSLRightInline}[1]{\parbox[t]{\linewidth - \csllabelwidth}{#1}\break}
\newcommand{\CSLIndent}[1]{\hspace{\cslhangindent}#1}
\ifLuaTeX
  \usepackage{selnolig}  % disable illegal ligatures
\fi

\title{Rehabilitation nach Tumorerkrankungen: Sekundärdatenanalyse zur
Rückkehr von Krebspatient:innen in den Beruf, 2010--2017}
\author{Johannes Soff}
\date{28 2 2022}

\begin{document}
\maketitle

\hypertarget{methode}{%
\section{Methode}\label{methode}}

\hypertarget{datengrundlage}{%
\subsection{Datengrundlage}\label{datengrundlage}}

Für die vorliegende Analyse wurde der Scientific Use File (SUF)
„Abgeschlossene Rehabilitationen 2010--2017'' der Deutschen
Rentenversicherung verwendet . Der Datenbestand basiert auf der
Reha-Statistik-Datenbasis-Verlaufserhebung nach § 13 Abs. 1 RSVwV und §
4 RSVwV der gesetzlichen Rentenversicherung und enthält personenbezogene
Daten mit Angaben zu den Versicherten und deren anspruchsberechtigten
Angehörigen. Eingeschlossen sind Personen mit einem der folgenden
Merkmale im Beobachtungszeitraum (2010--2017): mindestens eine
abgeschlossene Rehabilitationsleistung, bewilligte Rente oder
Zugehörigkeit zu einer bestimmten demografischen Kohorte (Tod vor oder
im Alter von 75 Jahren oder Zugehörigkeit zu einem bestimmten
Geburtsjahrgang). Personen, deren Anträge auf Rehabilitation oder eine
Rente endgültig abgelehnt wurden, sind ausgeschlossen. Für jedes
Rehabilitations-, Renten- oder demografische Ereignis werden die
einbezogenen Personen über einen Zeitraum von 8 Jahren beobachtet. Der
Datenbestand umfasst insgesamt eine disproportionale Zufallsstichprobe
von 20\% aller Versicherten (N ≈ 3,7 Millionen), wobei für die Analyse
der angegebene Gewichtungsfaktor verwendet wurde, um von der Stichprobe
auf die Zielpopulation zu schließen.

Da die DRV der Hauptzahler für Rehabilitationsleistungen Erwerbstätiger
in Deutschland ist, liegt die Altersspanne der Teilnehmer in diesem
Datensatz typischerweise zwischen 16 und 66 Jahren. Eine detaillierte
Beschreibung des Datensatzes, einschließlich des Stichprobendesigns,
kann dem Codeplan des Forschungsdatenzentrums entnommen werden
(Forschungsdatenzentrum der Rentenversicherung and DRV Bund 2020).

Aufgrund des Längsschnittdesigns mit Individualdaten von 2010--2017 ist
der SUF eine geeignete Quelle für die Analyse der Auswirkungen und
zeitlichen Trends von Rehabilitationsleistungen.

\hypertarget{teilnehmer-und-outcomes}{%
\subsubsection{Teilnehmer und Outcomes}\label{teilnehmer-und-outcomes}}

Aus dem vollständigen Datensatz von 3,7 Millionen Versicherten wurden
primär Daten über Leistungen zur medizinischen Rehabilitation für die
Auswertung verwendet. Im Anschluss wurden die entsprechenden Daten über
Leistungen zur Teilhabe am Arbeitsleben und dem Versicherungsverhältnis
sowie den geleisteten Beiträgen und Beitragszeigen ergänzt. Für diese
etwa 2,2 Millionen Patienten wurden die folgenden Ein- und
Ausschlusskriterien definiert.

\hypertarget{einschlusskriterien}{%
\subsubsection{Einschlusskriterien}\label{einschlusskriterien}}

In einem ersten Schritt wurden Rehabilitationspatienten mit mit der
Hauptdiagnose KHK oder ischämische Herzkrankheit (ICD10 I20.-I25.), die
eine medizinische Rehabilitation abgeschlossen hatten (Phase II). Im
nächsten Schritt haben wir die EBRP-III mit der unsere
Studienpopulation. Die Teilnahme musste innerhalb von sechs Monaten nach
Abschluss der Phase-II-Rehabilitation beginnen.

\begin{itemize}
\tightlist
\item
  Eingeschlossen wurden Patienten, die AUSWAHLKRITERIEN\_EINSCHLUSS
\item
  Dabei wurden Patienten ausgeschlossen, die
  AUSSCHLUSSKRITERIEN\_AUSSCHKUSS X (n=\ldots), Y (n=\ldots), Z
  (n=\ldots)
\item
  Deskriptive Tabelle
\end{itemize}

\hypertarget{tabelle-1}{%
\subsection{Tabelle 1}\label{tabelle-1}}

\hypertarget{datensatzbeschreibung}{%
\subsubsection{Datensatzbeschreibung}\label{datensatzbeschreibung}}

\begin{verbatim}
## Table printed with `knitr::kable()`, not {gt}. Learn why at
## https://www.danieldsjoberg.com/gtsummary/articles/rmarkdown.html
## To suppress this message, include `message = FALSE` in code chunk header.
\end{verbatim}

\begin{longtable}[]{@{}
  >{\raggedright\arraybackslash}p{(\columnwidth - 6\tabcolsep) * \real{0.6381}}
  >{\centering\arraybackslash}p{(\columnwidth - 6\tabcolsep) * \real{0.1143}}
  >{\centering\arraybackslash}p{(\columnwidth - 6\tabcolsep) * \real{0.1238}}
  >{\centering\arraybackslash}p{(\columnwidth - 6\tabcolsep) * \real{0.1238}}@{}}
\toprule
\begin{minipage}[b]{\linewidth}\raggedright
\textbf{Variable}
\end{minipage} & \begin{minipage}[b]{\linewidth}\centering
\textbf{Gesamt}, N = 83.538
\end{minipage} & \begin{minipage}[b]{\linewidth}\centering
\textbf{männlich}, N = 33.800
\end{minipage} & \begin{minipage}[b]{\linewidth}\centering
\textbf{weiblich}, N = 49.738
\end{minipage} \\
\midrule
\endhead
\textbf{Alter bei Beginn der ersten Rehabilitationsleistung, Mittelwert
(SD)} & 53,0 (8,1) & 54,1 (8,3) & 52,2 (7,8) \\
\textbf{Familienstand, n (\%)} & & & \\
ledig & 10.142 (12) & 4.418 (13) & 5.724 (12) \\
Verheiratet & 55.859 (67) & 24.413 (72) & 31.446 (63) \\
Geschieden & 8.902 (11) & 2.820 (8,3) & 6.082 (12) \\
Verwitwet & 2.146 (2,6) & 361 (1,1) & 1.785 (3,6) \\
entfällt & 6.489 (7,8) & 1.788 (5,3) & 4.701 (9,5) \\
\textbf{Medizinische Entlassungsdiagnose der Rehabilitation, n (\%)} & &
& \\
BN der Lippe, Mundhöhle, Pharynx (C00-14) & 3.146 (3,8) & 2.324 (6,9) &
822 (1,7) \\
BN Ösophagus, BN Magen (C15, C16) & 2.630 (3,1) & 1.827 (5,4) & 803
(1,6) \\
BN des Dünndarmes, Bösartige Neubildungen des Kolons (C17, C18) & 4.191
(5,0) & 2.320 (6,9) & 1.871 (3,8) \\
BN am Rektosigmoid, Übergang, BN des Rektums, BN des Anus und des
Analkanals (C19, C20, C21) & 3.699 (4,4) & 2.152 (6,4) & 1.547 (3,1) \\
BN Atmungsorgane (C30-39) & 5.152 (6,2) & 3.143 (9,3) & 2.009 (4,0) \\
BN Knochen (C40-41) & 226 (0,3) & 138 (0,4) & 88 (0,2) \\
Melanome der Haut (C43-44) & 1.603 (1,9) & 650 (1,9) & 953 (1,9) \\
Bildungen mesothelialen Gewebes u. des Weichteilgewebes (C45-49) & 707
(0,8) & 343 (1,0) & 364 (0,7) \\
BN Brustdrüse (C50) & 28.317 (34) & 101 (0,3) & 28.216 (57) \\
BN Vulva, Vagina, Cervix uteri (C51-53) & 2.074 (2,5) & 0 (0) & 2.074
(4,2) \\
BN Uterus (C54) & 1.900 (2,3) & 0 (0) & 1.900 (3,8) \\
BN Plazenta (C56-58) & 2.131 (2,6) & 0 (0) & 2.131 (4,3) \\
Bösartige Neubildungen der männlichen Genitalorgane (C60-63) & 10.834
(13) & 10.834 (32) & 0 (0) \\
Bösartige Neubildungen der Harnorgane (C64-68) & 5.851 (7,0) & 4.182
(12) & 1.669 (3,4) \\
BN des Auges, des Gehirns und sonstiger Teile des ZNS (C69-72) & 1.112
(1,3) & 684 (2,0) & 428 (0,9) \\
BN Schilddrüse, und andere endokrine Drüsen (C73-75) & 1.499 (1,8) & 417
(1,2) & 1.082 (2,2) \\
BN Leber, Gallengänge /-blase, Pankreas, sonstige
Verdauungsorgange,ungenauer, sekund.+ n.n. bez./ an mehr. Lokalisa. (C &
2.680 (3,2) & 1.354 (4,0) & 1.326 (2,7) \\
Hodgkin-Krankheit (C81) & 1.191 (1,4) & 693 (2,1) & 498 (1,0) \\
Non-Hodgkin-Krankheiten (C82-90) & 3.332 (4,0) & 1.878 (5,6) & 1.454
(2,9) \\
Lymphatische, myeloische und sonstige Leukämie (C91-96) & 1.263 (1,5) &
760 (2,2) & 503 (1,0) \\
\textbf{Psychische oder Verhaltensstörungen nach ICD-10 als weitere
medizinische Entlassungsdiagnose, n (\%)} & 16.580 (20) & 5.255 (16) &
11.325 (23) \\
\textbf{1. Diagnose: Behandlungsergebnis, n (\%)} & & & \\
Schlüsselziffer 1-3 trifft nicht zu & 11.066 (13) & 3.712 (11) & 7.354
(15) \\
Gebessert & 59.724 (71) & 24.729 (73) & 34.995 (70) \\
Unverändert & 12.623 (15) & 5.301 (16) & 7.322 (15) \\
Verschlechtert & 125 (0,1) & 58 (0,2) & 67 (0,1) \\
\textbf{Teilrentenkennzeichen, n (\%)} & & & \\
kein Rentenbezug/Angehöriger & 59.708 (71) & 23.089 (68) & 36.619
(74) \\
Vollrente & 16.000 (19) & 7.657 (23) & 8.343 (17) \\
1/3 Teilrente (Fälle bis 30.06.2017) & 14 (\textless0,1) & 8
(\textless0,1) & 6 (\textless0,1) \\
1/2 Teilrente (Fälle bis 30.06.2017) & 10 (\textless0,1) & 4
(\textless0,1) & 6 (\textless0,1) \\
2/3 Teilrente (Fälle bis 30.06.2017) & 17 (\textless0,1) & 9
(\textless0,1) & 8 (\textless0,1) \\
Teilrente & 30 (\textless0,1) & 16 (\textless0,1) & 14 (\textless0,1) \\
Vollrente wegen Erwerbsminderung & 7.290 (8,7) & 2.844 (8,4) & 4.446
(8,9) \\
Rente in Höhe einer 1/3 BU-Rente (Fälle bis 30.06.2017) & 0 (0) & 0 (0)
& 0 (0) \\
Rente wg. voller/teilw. EM in Höhe von 1/2 (Fälle bis 30.06.2017) & 94
(0,1) & 27 (\textless0,1) & 67 (0,1) \\
Rente in Höhe einer 2/3 BU-Rente (Fälle bis 30.06.2017) & 0 (0) & 0 (0)
& 0 (0) \\
EU-Rente in Höhe einer vollen BU-Rente (Fälle bis 30.06.2017) & 0 (0) &
0 (0) & 0 (0) \\
Rente ruht in voller Höhe & 288 (0,3) & 106 (0,3) & 182 (0,4) \\
Rente wg. voller EM in Höhe von 1/4 (Fälle bis 30.06.2017) & 16
(\textless0,1) & 9 (\textless0,1) & 7 (\textless0,1) \\
Rente wg. voller EM in Höhe von 3/4 (Fälle bis 30.06.2017) & 49
(\textless0,1) & 23 (\textless0,1) & 26 (\textless0,1) \\
Teilrente wegen Erwerbsminderung & 22 (\textless0,1) & 8 (\textless0,1)
& 14 (\textless0,1) \\
\textbf{Zeitraum vom Beginn der ersten medizinischen Rehabilitation bis
zum Rentenbeginn in Monaten, Mittelwert (SD)} & 27 (23) & 26 (23) & 27
(24) \\
\textbf{Versterben im Beobachtungszeitraum, n (\%)} & 14.887 (18) &
7.372 (22) & 7.515 (15) \\
\textbf{Zeitraum vom Beginn der ersten medizinischen Rehabilitation bis
zum Todeszeitpunkt in Monaten, Mittelwert (SD)} & 28 (22) & 26 (22) & 30
(22) \\
\textbf{Schulbildung, n (\%)} & & & \\
entfällt/keine Aussage möglich/Angehöriger & 24.713 (30) & 11.258 (33) &
13.455 (27) \\
ohne Schulabschluss & 519 (0,6) & 285 (0,8) & 234 (0,5) \\
Haupt-/Volksschulabschluss & 11.625 (14) & 5.713 (17) & 5.912 (12) \\
Mittlere Reife oder gleichwertiger Abschluss & 18.355 (22) & 5.748 (17)
& 12.607 (25) \\
Abitur/Fachabitur & 11.348 (14) & 4.580 (14) & 6.768 (14) \\
Abschluss unbekannt & 16.978 (20) & 6.216 (18) & 10.762 (22) \\
\textbf{Sozialversicherungspflichtige Beschäftigung, n (\%)} & & & \\
Beschäftigung besteht nicht & 64.950 (78) & 26.048 (77) & 38.902 (78) \\
Beschäftigung besteht & 18.588 (22) & 7.752 (23) & 10.836 (22) \\
\textbf{Berufsgruppenklassifikation nach dem Statistikband zur
Rehabilitation der Rentenversicherung (Grundlage ist die KldB 88), n
(\%)} & & & \\
keine/falsche DEÜV-Meldung & 102 (0,1) & 58 (0,2) & 44 (\textless0,1) \\
ohne Beruf & 11.634 (14) & 4.755 (14) & 6.879 (14) \\
Landwirt. Berufe & 990 (1,2) & 572 (1,7) & 418 (0,8) \\
Bergleute, Mineralgewinner & 29 (\textless0,1) & 23 (\textless0,1) & 6
(\textless0,1) \\
Herstellerberufe & 1.639 (2,0) & 1.074 (3,2) & 565 (1,1) \\
Metallberufe & 7.454 (8,9) & 6.061 (18) & 1.393 (2,8) \\
Textilberufe & 451 (0,5) & 182 (0,5) & 269 (0,5) \\
Ernährungberufe & 2.053 (2,5) & 767 (2,3) & 1.286 (2,6) \\
Bauberufe & 5.264 (6,3) & 4.038 (12) & 1.226 (2,5) \\
Technische Berufe & 4.393 (5,3) & 2.576 (7,6) & 1.817 (3,7) \\
Handels- u. Verkehrsberufe & 16.521 (20) & 6.780 (20) & 9.741 (20) \\
Verwaltungs-, Organisationsberufe & 15.673 (19) & 4.480 (13) & 11.193
(23) \\
Gesundheitsberufe & 8.319 (10,0) & 860 (2,5) & 7.459 (15) \\
Lehrberufe etc. & 1.954 (2,3) & 492 (1,5) & 1.462 (2,9) \\
sonst. Dienstleistungsberufe & 6.730 (8,1) & 944 (2,8) & 5.786 (12) \\
Sonstige & 332 (0,4) & 138 (0,4) & 194 (0,4) \\
\textbf{Erwerbsstatus u. -umfang vor Antragsstellung, n (\%)} & & & \\
nicht erwerbstätig / nicht Hausmann/frau / nicht arbeitslos oder
Präventionsleistung & 11.461 (14) & 3.347 (9,9) & 8.114 (16) \\
Ganztagsarbeit o. Wechselschicht & 42.231 (51) & 22.543 (67) & 19.688
(40) \\
Ganztagsarbeit m. Wechselschicht & 6.646 (8,0) & 3.496 (10) & 3.150
(6,3) \\
Ganztagsarbeit m. Nachtschicht & 2.002 (2,4) & 1.232 (3,6) & 770
(1,5) \\
Teilzeitarbeit/ weniger als Hälfte d.~üblich. Arbeitszeit & 3.173 (3,8)
& 204 (0,6) & 2.969 (6,0) \\
Teilzeitarbeit/ mind. Hälfte d.~üblich. Arbeitszeit & 11.558 (14) & 604
(1,8) & 10.954 (22) \\
Hausmann/frau & 1.785 (2,1) & 50 (0,1) & 1.735 (3,5) \\
arbeitslos & 4.581 (5,5) & 2.291 (6,8) & 2.290 (4,6) \\
Sonstige & 101 (0,1) & 33 (\textless0,1) & 68 (0,1) \\
\textbf{Arbeitsunfähigkeit in den letzten 12 Monaten, n (\%)} & & & \\
Keine Arbeitsunfähigkeit & 6.554 (7,8) & 2.401 (7,1) & 4.153 (8,3) \\
unter 3 Monate & 21.319 (26) & 12.952 (38) & 8.367 (17) \\
3 bis 6 Monate & 16.127 (19) & 6.108 (18) & 10.019 (20) \\
6 und mehr Monate & 32.772 (39) & 11.093 (33) & 21.679 (44) \\
Nicht erwerbstätig & 6.766 (8,1) & 1.246 (3,7) & 5.520 (11) \\
\textbf{Arbeitsfähigkeit, n (\%)} & & & \\
keine Aussage möglich & 369 (0,4) & 171 (0,5) & 198 (0,4) \\
Arbeitsfähig & 15.178 (18) & 4.929 (15) & 10.249 (21) \\
Arbeitsunfähig & 63.141 (76) & 28.051 (83) & 35.090 (71) \\
Beurteilung nicht erforderlich & 4.850 (5,8) & 649 (1,9) & 4.201
(8,4) \\
\textbf{Leistungsfähigkeit im letzten Beruf, n (\%)} & & & \\
keine Angabe erforderlich, trifft nicht zu & 1.040 (1,2) & 390 (1,2) &
650 (1,3) \\
6 Stunden und mehr & 64.888 (78) & 25.164 (74) & 39.724 (80) \\
3 bis unter 6 Stunden & 3.220 (3,9) & 800 (2,4) & 2.420 (4,9) \\
unter 3 Stunden & 14.390 (17) & 7.446 (22) & 6.944 (14) \\
\textbf{Leistungsfähigkeit andere Tätigkeit, n (\%)} & & & \\
keine Angabe erforderlich, trifft nicht zu & 1.132 (1,4) & 424 (1,3) &
708 (1,4) \\
6 Stunden und mehr & 72.534 (87) & 28.922 (86) & 43.612 (88) \\
3 bis unter 6 Stunden & 2.183 (2,6) & 614 (1,8) & 1.569 (3,2) \\
unter 3 Stunden & 7.689 (9,2) & 3.840 (11) & 3.849 (7,7) \\
\textbf{Stellung im Beruf, n (\%)} & & & \\
Nicht erwerbstätig oder Präventionsleistung & 11.063 (13) & 2.379 (7,0)
& 8.684 (17) \\
Auszubildender & 365 (0,4) & 178 (0,5) & 187 (0,4) \\
Ungelernter Arbeiter & 4.299 (5,1) & 1.477 (4,4) & 2.822 (5,7) \\
Angelernter Arbeiter & 3.621 (4,3) & 1.726 (5,1) & 1.895 (3,8) \\
Facharbeiter & 17.692 (21) & 12.027 (36) & 5.665 (11) \\
Meister, Polier & 503 (0,6) & 430 (1,3) & 73 (0,1) \\
Angestellter & 42.779 (51) & 13.787 (41) & 28.992 (58) \\
Beamter & 26 (\textless0,1) & 11 (\textless0,1) & 15 (\textless0,1) \\
Selbstständiger & 3.190 (3,8) & 1.785 (5,3) & 1.405 (2,8) \\
\textbf{Anschlussrehabilitation, n (\%)} & & & \\
trifft nicht zu & 31.047 (37) & 10.911 (32) & 20.136 (40) \\
trifft zu & 52.491 (63) & 22.889 (68) & 29.602 (60) \\
\textbf{Aufforderung d.~Krankenkasse, n (\%)} & & & \\
trifft nicht zu & 81.255 (97) & 32.482 (96) & 48.773 (98) \\
trifft zu & 2.283 (2,7) & 1.318 (3,9) & 965 (1,9) \\
\textbf{Zeitraum von der Antragstellung eines Antrags auf medizinische
Rehabilitation bis zu seiner Bewilligung in Tagen, Mittelwert (SD)} & 26
(31) & 20 (26) & 29 (33) \\
\textbf{Stationäre Behandlung/Operation, n (\%)} & & & \\
Nein & 81.802 (98) & 32.790 (97) & 49.012 (99) \\
Ja & 1.736 (2,1) & 1.010 (3,0) & 726 (1,5) \\
\textbf{Psych. Beratung/Psychotherapie, n (\%)} & & & \\
Nein & 74.562 (89) & 31.860 (94) & 42.702 (86) \\
Ja & 8.976 (11) & 1.940 (5,7) & 7.036 (14) \\
\textbf{Heil-und Hilfsmittel inkl. Physioth./Ergoth., n (\%)} & & & \\
Nein & 68.188 (82) & 28.673 (85) & 39.515 (79) \\
Ja & 15.350 (18) & 5.127 (15) & 10.223 (21) \\
\textbf{Stufenweise Wiedereingliederung, n (\%)} & & & \\
Nein & 77.629 (93) & 31.801 (94) & 45.828 (92) \\
Ja & 5.909 (7,1) & 1.999 (5,9) & 3.910 (7,9) \\
\textbf{Leistungen zur Teilhabe am Arbeitsleben, n (\%)} & & & \\
Nein & 78.283 (94) & 31.547 (93) & 46.736 (94) \\
Ja & 5.255 (6,3) & 2.253 (6,7) & 3.002 (6,0) \\
\textbf{Rehabilitationssport, n (\%)} & & & \\
Nein & 70.847 (85) & 31.161 (92) & 39.686 (80) \\
Ja & 12.691 (15) & 2.639 (7,8) & 10.052 (20) \\
\textbf{Funktionstraining, n (\%)} & & & \\
Nein & 83.051 (99) & 33.594 (99) & 49.457 (99) \\
Ja & 487 (0,6) & 206 (0,6) & 281 (0,6) \\
\textbf{Reha-Nachsorge, n (\%)} & & & \\
Nein & 82.943 (99) & 33.504 (99) & 49.439 (99) \\
Ja & 595 (0,7) & 296 (0,9) & 299 (0,6) \\
\textbf{Selbsthilfegruppe, n (\%)} & & & \\
Nein & 75.377 (90) & 30.757 (91) & 44.620 (90) \\
Ja & 8.161 (9,8) & 3.043 (9,0) & 5.118 (10) \\
\textbf{Körperliche Arbeitsschwere, n (\%)} & & & \\
keine Angabe & 5.249 (6,3) & 2.468 (7,3) & 2.781 (5,6) \\
schwere Arbeiten & 1.786 (2,1) & 1.482 (4,4) & 304 (0,6) \\
mittelschwere Arbeiten & 12.039 (14) & 7.186 (21) & 4.853 (9,8) \\
leichte bis mittelschwere Arbeiten & 42.192 (51) & 14.804 (44) & 27.388
(55) \\
leichte Arbeiten & 22.272 (27) & 7.860 (23) & 14.412 (29) \\
\textbf{Arbeitshaltung im Stehen, n (\%)} & & & \\
keine Angabe & 6.059 (7,3) & 2.943 (8,7) & 3.116 (6,3) \\
ständig & 13.734 (16) & 5.236 (15) & 8.498 (17) \\
überwiegend & 39.877 (48) & 16.267 (48) & 23.610 (47) \\
zeitweise & 23.868 (29) & 9.354 (28) & 14.514 (29) \\
\textbf{Arbeitshaltung im Gehen, n (\%)} & & & \\
keine Angabe & 6.079 (7,3) & 2.951 (8,7) & 3.128 (6,3) \\
ständig & 14.206 (17) & 5.219 (15) & 8.987 (18) \\
überwiegend & 40.772 (49) & 16.066 (48) & 24.706 (50) \\
zeitweise & 22.481 (27) & 9.564 (28) & 12.917 (26) \\
\textbf{Arbeitshaltung im Sitzen, n (\%)} & & & \\
keine Angabe & 6.032 (7,2) & 2.915 (8,6) & 3.117 (6,3) \\
ständig & 30.335 (36) & 11.522 (34) & 18.813 (38) \\
überwiegend & 39.979 (48) & 15.918 (47) & 24.061 (48) \\
zeitweise & 7.192 (8,6) & 3.445 (10) & 3.747 (7,5) \\
\textbf{Tagesschicht, n (\%)} & & & \\
Nein & 8.103 (9,7) & 3.873 (11) & 4.230 (8,5) \\
Ja & 75.435 (90) & 29.927 (89) & 45.508 (91) \\
\textbf{Früh-/Spätschicht, n (\%)} & & & \\
Nein & 28.992 (35) & 12.647 (37) & 16.345 (33) \\
Ja & 54.546 (65) & 21.153 (63) & 33.393 (67) \\
\textbf{Nachtschicht, n (\%)} & & & \\
Nein & 61.000 (73) & 23.614 (70) & 37.386 (75) \\
Ja & 22.538 (27) & 10.186 (30) & 12.352 (25) \\
\textbf{Wohnort: altes/neues Bundesgebiet, n (\%)} & & & \\
fehlende Angabe & 1.488 (1,8) & 496 (1,5) & 992 (2,0) \\
West (inkl. West-Berlin, inkl. Ausland) & 63.132 (76) & 25.165 (74) &
37.967 (76) \\
Ost (inkl. Ost-Berlin) & 18.918 (23) & 8.139 (24) & 10.779 (22) \\
\textbf{Siedlungsstruktureller Kreistyp, n (\%)} & & & \\
unbekannt & 2.029 (2,4) & 691 (2,0) & 1.338 (2,7) \\
Kreisfreie Großstädte & 20.419 (24) & 7.896 (23) & 12.523 (25) \\
Städtische Kreise & 31.896 (38) & 13.012 (38) & 18.884 (38) \\
Ländliche Kreise mit Verdichtungsansätzen & 15.656 (19) & 6.569 (19) &
9.087 (18) \\
Dünn besiedelte ländliche Kreise & 13.538 (16) & 5.632 (17) & 7.906
(16) \\
\textbf{Bundesland des Maßnahmeortes, n (\%)} & & & \\
fehlende Angabe & 3.725 (4,5) & 1.303 (3,9) & 2.422 (4,9) \\
Schleswig-Holstein & 8.070 (9,7) & 2.098 (6,2) & 5.972 (12) \\
Hamburg & 12 (\textless0,1) & 7 (\textless0,1) & 5 (\textless0,1) \\
Niedersachsen & 5.147 (6,2) & 2.236 (6,6) & 2.911 (5,9) \\
Bremen & 2 (\textless0,1) & 2 (\textless0,1) & 0 (0) \\
Nordrhein-Westfalen & 7.909 (9,5) & 3.728 (11) & 4.181 (8,4) \\
Hessen & 10.130 (12) & 5.192 (15) & 4.938 (9,9) \\
Reinland-Pfalz & 2.966 (3,6) & 939 (2,8) & 2.027 (4,1) \\
Baden-Würtemberg & 14.477 (17) & 5.418 (16) & 9.059 (18) \\
Bayern & 10.453 (13) & 4.159 (12) & 6.294 (13) \\
Saarland & 585 (0,7) & 291 (0,9) & 294 (0,6) \\
Brandenburg & 2.666 (3,2) & 1.217 (3,6) & 1.449 (2,9) \\
Mecklenburg-Vorpommern & 6.425 (7,7) & 2.053 (6,1) & 4.372 (8,8) \\
Sachsen & 4.726 (5,7) & 2.348 (6,9) & 2.378 (4,8) \\
Sachsen-Anhalt & 2.633 (3,2) & 1.055 (3,1) & 1.578 (3,2) \\
Thüringen & 3.270 (3,9) & 1.578 (4,7) & 1.692 (3,4) \\
Ausland & 13 (\textless0,1) & 6 (\textless0,1) & 7 (\textless0,1) \\
Berlin West & 290 (0,3) & 138 (0,4) & 152 (0,3) \\
Berlin Ost & 39 (\textless0,1) & 32 (\textless0,1) & 7 (\textless0,1) \\
\bottomrule
\end{longtable}

Variablenauswahl treffen

\hypertarget{refs}{}
\begin{CSLReferences}{1}{0}
\leavevmode\vadjust pre{\hypertarget{ref-forschungsdatenzentrumderrentenversicherung2020}{}}%
Forschungsdatenzentrum der Rentenversicherung, and DRV Bund. 2020.
{``Scientific Use File. Abgeschlossene Rehabilitationen 2010-2017 Im
Versicherungsverlauf (Sufrsdlv17b).''} Berlin.

\end{CSLReferences}

\end{document}
